\section{Basic Concepts:}

\subsection{Counting Number of Parenthesizations:}

Before solving the matrix-chain multiplication problem by dynamic programming, let us convince ourselves that exhaustively checking all possible parenthesizations does not yield an efficient algorithm. Denote the number of alternative parenthesizations of a sequence of {\bfseries n} matrices by {\bfseries P(n)}. When {\itshape n = 1}, we have just one matrix and therefore only one way to fully parenthesize the matrix product. When $n \geq 2$, a fully parenthesized matrix product is the product of two fully parenthesized matrix sub-products, and the split between the two sub-products may occur between the k-th and ( k + 1 )st matrices for any {\itshape k = 1, 2, ..., n - 1}. Thus, we obtain the recurrence: \hfill \break

\begin{ceqn}
\begin{align}
P(\ n\ ) = \left\{
\begin{array}{ll}
1 & \mathrm {if\ } n\ =\ 1, \\
\sum_{k = 1}^{n - 1}\ P(\ k\ )P(\ n\ -\ k\ ) & \mathrm {if\ } n\ \geq\ 2.\\
\end{array}
\right.
\end{align}
\end{ceqn}

\subsection{Dynamic Programming:}

We shall use the dynamic-programming method to determine how to optimally parenthesize a matrix chain. In so doing, we shall follow the four-step sequence:

\begin{itemize}
\item Characterize the structure of an optimal solution.
\item Recursively define the value of an optimal solution.
\item Compute the value of an optimal solution.
\item Construct an optimal solution from computed information.
\end{itemize}

\pagebreak