\subsection{Hamiltonian-Cycle Verification:}

As we have seen in section 1, the {\bfseries Hamiltonian-Cycle Problem} is {\itshape NP-complete}, but if we suppose that a graph {\itshape G} it's {\itshape Hamiltonian} and we have a {\itshape Certificate C} ( list of vertices in order along the {\itshape Hamiltonian-Cycle} ) that can prove this supposition, then the problem can be solvable in polynomial time only by checking wheter it is a permutation of the vertices of {\itshape V} and whether each of the consecutive edges along the cycle actually exists in the graph. In theory, the algorithm should run in {\bfseries $\mathcal{O}(n^{2})$}. \hfill \break

The algorithm that we wrote it's in an only {\itshape python module} that has 3 methods including the main. \hfill \break

\begin{lstlisting}
def main ( ):
    graph = { 1: [ 2, 3, 4 ], 2: [ 1 ], 3: [ 1, 2, 4 ], 4: [ 1, 2, 3 ] }
    certificate = [ 2, 1, 3, 4, 2 ]
    verify_hamiltonian ( graph, certificate )
\end{lstlisting} \hfill \break

As we can see, the main method only declares a {\itshape graph ( G )} and the {\itshape certificate ( C )}, and in line 4 calls the method {\itshape verify$\_$hamiltonian ( G, C )}. \hfill \break

\begin{lstlisting}
def verify_hamiltonian ( graph, certificate ):
    for i in range ( len ( graph ) ):
        neighbors = graph [ certificate [ i ] ]
        if ( certificate [ i + 1 ] not in neighbors ):
            printer ( -1, certificate )
    printer ( 0, certificate )
\end{lstlisting} \hfill \break

The previous algorithm verify if {\itshape C} it's or not a {\bfseries Hamiltonian-Cycle} of {\itshape G}. The algorithm works as follows, in the input has the {\itshape graph} and the {\itshape certificate} the in line 2 there is a {\bfseries for} loop that will run from {\itshape 0} to {\itshape n - 1} where {\itshape n} it's the number of vertices in {\itshape G}, then in line 3 we will store the {\itshape neighbors} or {\itshape adjacent} vertices of the node in which the algorithm it's situated. In line 4 the program asks if the next vertex in the given {\itshape certificate} it's or not a neighbor of the current node, in case that it is, the program continues running, otherwise calls the method printer with an error flag ( -1 ) and exits the program. \hfill \break

Finally the last method prints in console if {\itshape C} it's a {\bfseries Hamiltonian-Cycle} of {\itshape G}. \hfill \break

\begin{lstlisting}
def printer ( flag, C ):
    if ( flag == -1 ):
        print ( "\n\n\tThe certificate {} isn't a Hamiltonian Cycle.\n".format ( C ) )
        exit ( 0 )
    print ( "\n\n\tThe certificate {} is a Hamiltonian Cycle.\n".format ( C ) )
\end{lstlisting}

\pagebreak