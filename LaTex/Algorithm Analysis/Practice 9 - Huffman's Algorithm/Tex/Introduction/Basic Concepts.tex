\section{Basic Concepts:}

A greedy algorithm obtains an optimal solution to a problem by making a sequence of choices. At each decision point, the algorithm makes choice that seems best at the moment. This heuristic strategy does not always produce an optimal solution, but as we saw in the activity-selection problem explained by {\itshape Team 2}, sometimes it does. The following steps help to develop a greedy algorithm.

\begin{itemize}
\item Determine the optimal substructure of the problem.
\item Develop a recursive solution.
\item Show that if we make the greedy choice, then only one sub-problem remains.
\item Prove that it is always safe to make the greedy choice. ( Steps 3 and 4 can occur in either order.)
\item Develop a recursive algorithm that implements the greedy strategy.
\item Convert the recursive algorithm to an iterative algorithm.
\end{itemize}

\pagebreak