\subsection{Function B:}

Calculate the complexity order of the following algorithm in the best $(\Omega)$ and the worst $(O)$
case (it's not necessary to do the analysis line-per-line, in this case, you can apply
properties of the algorithms ):\hfill \break

{{\bfseries\color{Violet}{Function B}}  ( $A[0,...,n-1],\ x\ integer$ ):

\begin{lstlisting}
	for i = 0 i < n do 
		if(A[i] < x)
			A[i] = min(A[0,...,n-1])
		else if (A [i] > x)
			A[i] = max(A[0,...,n-1])
		else
			exit
\end{lstlisting} \hfill

\begin{itemize}
\item {\bfseries\itshape\color{Maroon}{Demonstration:}} 
\end{itemize} 

{\bfseries\itshape\color{armygreen}{Observation:}} {\itshape\color{armygreen}{Best case: Let 'x' be in the first position of A.}} 

\begin{itemize}
\item {\bfseries\itshape\color{Violet}{Analyzing the complexity of each line:}}
\begin{enumerate}
\item Line 1 = $\theta\ (\ 1\ )$.
\begin{tasks}
\task Line 2 = $\theta\ (\ 1\ )$.
\task Line 4 = $\theta\ (\ 1\ )$.
\task Line 6 = $\theta\ (\ 1\ )$.
\task Line 7 = $\theta\ (\ 1\ )$.
\end{tasks}
\end{enumerate}
\end{itemize} 

{\bfseries\itshape\color{armygreen}{Observation:}} {\itshape\color{armygreen}{The lines 3 and 5 doesn't count because 'x' doesn't fulfill the condition.}} 

\begin{itemize}
\item {\bfseries\itshape\color{Violet}{Then, from all lines we can conclude:}}
\end{itemize} \hfill

\begin{ceqn}
\begin{align}
T\ (\ n\ )\ =\ \theta\ (\ 1\ )\
\end{align}
\end{ceqn} \hfill

\begin{itemize}
\item {\bfseries\itshape\color{Violet}{Finally:}}
\end{itemize} \hfill

\begin{ceqn}
\begin{align}
Function\ B\ \in\ \Omega\ (\ 1\ )
\end{align}
\end{ceqn}
\pagebreak

\pagebreak

\begin{itemize}
\item {\bfseries\itshape\color{Maroon}{Demonstration:}} 
\end{itemize} 

{\bfseries\itshape\color{armygreen}{Observation:}} {\itshape\color{armygreen}{Worst case: Let 'x' be in the position [ n - 1 ] or isn't be in A.}} 

\begin{multicols}{2}
\begin{itemize}
\item {\bfseries\itshape\color{Violet}{Analyzing the complexity of each line:}}
\begin{enumerate}

\item Line 1 = $\theta\ (\ n\ )$.
\begin{tasks}
\task Line 2, 3 = $\theta\ (\ n_{1}\ )$.
\task Line 4, 5 = $\theta\ (\ n_{2}\ )$.
\task Line 6, 7 = $\theta\ (\ 1\ )$.
\end{tasks}
\end{enumerate}
\hfill \break \break \break \break \break  
\end{itemize} 
\end{multicols} \hfill

{\bfseries\itshape\color{armygreen}{Observation:}} {\itshape\color{armygreen}{Line 6 and 7 are $\theta ( 1 )$ if 'x' it's in the position [ n - 1 ].}} 

\begin{itemize}
\item {\bfseries\itshape\color{Violet}{Then, from all lines:}}
\end{itemize} \hfill

\begin{ceqn}
\begin{align}
T\ (\ n\ )\ =\ \theta\ (\ n\ (\ n_{1}\ +\ n_{2}\ +\ 1\ )\ )\ =\ \theta\ (\ n^{2}\ )\
\end{align}
\end{ceqn} \hfill

\begin{itemize}
\item {\bfseries\itshape\color{Violet}{Finally:}}
\end{itemize} \hfill

\begin{ceqn}
\begin{align}
Function\ B\ \in\ O\ (\ n^{2}\ )
\end{align}
\end{ceqn}

\pagebreak