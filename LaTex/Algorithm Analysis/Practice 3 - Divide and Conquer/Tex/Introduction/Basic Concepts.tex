\section{Basic Concepts:}

\subsection{Divide-and-Conquer Paradigm:}

The divide-and-conquer paradigm involves three steps at each level of the recursion:

\begin{itemize}
\item {\bfseries\itshape Divide:} Divide the problem into a number of sub-problems that are smaller instances of the same problem.
\item {\bfseries\itshape Conquer:} Conquer the sub-problems by solving them recursively. If the sub-problem sizes are small enough, however, just solve the sub-problems in a straightforward manner.
\item {\bfseries\itshape Combine:} Combine the solutions to the sub-problems into the solution for the original problem.
\end{itemize}

\subsection{Merge-Sort Algorithm:}

The {\bfseries\itshape merge sort algorithm} closely follows the divide-and-conquer paradigm. In- tuitively, it operates as follows.

\begin{tasks}
\task {\bfseries\itshape Divide:} Divide the n-element sequence to be sorted into two sub-sequences of n=2 elements each.
\task {\bfseries\itshape Conquer:} Sort the two sub-sequences recursively using merge sort.
\task {\bfseries\itshape Combine:} Merge the two sorted sub-sequences to produce the sorted answer.
\end{tasks}

The recursion “bottoms out” when the sequence to be sorted has length 1, in which case there is no work to be done, since every sequence of length 1 is already in sorted order. The key operation of the merge sort algorithm is the merging of two sorted sequences in the “combine” step. 

\pagebreak