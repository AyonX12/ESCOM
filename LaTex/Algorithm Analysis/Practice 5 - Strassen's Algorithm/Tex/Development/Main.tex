\subsection{Main.py}

This module has the main implementation, as we can see in the code bellow we only call 3 methods:

\begin{itemize}
\item create ( ): Create and return the matrices A and B.
\item strassen ( ): Implements the strassen algorithm, return the resulting matrix and receive as parameters the matrices A and B.
\item printer ( ): This method it's nested in this module, prints on screen the matrices A, B and C.
\end{itemize}

\begin{lstlisting}
if ( __name__ == "__main__" ):
    A, B = create ( )
    C = strassen ( A, B )
    printer ( A, B, C )
\end{lstlisting} \hfill

The module as we mention, has another method, {\bfseries\itshape printer ( A, B, C )} gives format to the matrices and display them in console. \hfill \break

\begin{lstlisting}
def printer ( A, B, C ):
    assert len ( A ) == len ( B ) == len ( C )
    print ( "\n\tStrassen Algorithm:" )
    print ( "\n\tMatrix A:\n" )
    list ( map ( lambda x: print ( "\t{}".format ( x ) ), A ) )
    print ( "\n\tMatrix B:\n" )
    list ( map ( lambda x: print ( "\t{}".format ( x ) ), B ) )
    print ( "\n\tProduct A * B:\n" )
    list ( map ( lambda x: print ( "\t{}".format ( x ) ), C ) )
    n = len ( C )
    print ( "\n\tWhere: A, B, C ∈ M [ {}x{} ] ( ℤ+ )\n".format ( n, n ) )
\end{lstlisting} \hfill

{\bfseries\itshape\color{carmine}{Observation:}} {\itshape\color{carmine}{In line 2 of the code above, the keyword {\bfseries assert} validates that both A, B and the product C has the same size, in case that the condition doesn't satisfy the program will stop the execution and throw an error.}}

\pagebreak