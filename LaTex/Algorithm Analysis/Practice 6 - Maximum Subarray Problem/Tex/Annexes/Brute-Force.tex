\subsection{Maximum Subarray-Brute Force}

Proposed algorithm for maximum subarray using Brute-Force. \hfill \break

{\bfseries\color{Violet}{function}}  BRUTE-FORCE-MAXIMUM-SUBARRAY (A)
\begin{lstlisting}[mathescape=true]
max = -$\infty$
for i = 0 to A.lenght
	sum = 0
	for j = i to A.lenght
		sum += A[j]
		if ( sum > max )
			max = sum 
return max
\end{lstlisting} \hfill

\begin{itemize}
\item {\bfseries\itshape\color{carmine}{Demonstration:}} 
\end{itemize} 

\begin{itemize}
\item {\bfseries\itshape\color{Violet}{Analyzing the complexity of each line:}}
\begin{enumerate}
\item Line 1 = $\theta\ (\ 1\ )$.
\item Line 2 = $\theta\ (\ n\ )$.
\begin{tasks}
\task line 3 = $\theta\ (\ 1\ )$.
\task line 4 = $\theta\ (\ n\ )$.
\task line 5, 6, 7 = $\theta\ (\ 1\ )$.
\end{tasks}
\item Line 8 = $\theta\ (\ 1\ )$.  
\end{enumerate}
\end{itemize} \hfill

\begin{itemize}
\item {\bfseries\itshape\color{Violet}{Then, from all lines we can conclude:}}
\end{itemize} \hfill

\begin{ceqn}
\begin{align}
T\ (\ n\ )\ =\ \theta\ (\ n\ (\ 1\ +\ n\ +\ 1\ )\ )\ =\ \theta\ (\ n^{2}\ )\
\end{align}
\end{ceqn} \hfill

\begin{itemize}
\item {\bfseries\itshape\color{Violet}{Finally:}}
\end{itemize} \hfill

\begin{ceqn}
\begin{align}
 Brute-Force\ Maximum\ Subarray\  \in\ O\ (\ n^{2}\ )
\end{align}
\end{ceqn}

\pagebreak