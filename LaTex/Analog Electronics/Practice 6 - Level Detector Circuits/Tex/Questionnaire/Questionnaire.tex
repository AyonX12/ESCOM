\section{Questionnaire:}

\begin{itemize} 
\item {\bfseries What represents the negative sign in the circuits: Inverter, adder, derivator and integrator?} \hfill \break

The negative sign of the expression indicates the phase inversion between the input and the output. \hfill \break

\item {\bfseries Explain because it exists a difference between the voltage of theoretical and practical exit of the circuits, adder and subtract:} \hfill \break

The difference between the values occurs when considering that the operational amplifier is ideal because of the virtual ground that we generate in the moment of realizing the calculations. \hfill \break

\item {\bfseries Which function has the circuit follower of voltage?} \hfill \break

At first glance it seems that the voltage follower, having a unitary closed loop gain, would have no interest from the electronic point of view. However, having a zero-input current (Is = 0) allows us to couple a voltage source with relatively high input resistance to a load with relatively low resistance, without the charging effect occurring. It is said that the voltage follower produces an electrical insulation between the source and the load. For this reason, the voltage follower is also called a separator or buffer. \hfill \break

\item {\bfseries Which is the purpose to add him a resistance in parallel to the capacitor in the integrador and a capacitor in parallel to the resistance of the derivator?} \hfill \break

\begin{tasks}
\task To prevent the occurrence of any continuous voltage (eg imperfections of the AO) at the input of the Integrator to bring its output to saturation, an RC resistance is placed in parallel with the capacitor to limit the gain in DC to -RC / R. \hfill \break 
\task If there was noise at the input, this would normally be of a higher frequency compared to the signal to be derived, this would cause smaller noise values ​​to appear at the output much larger. To avoid this, a resistor R1 (in series with the capacitor C) is placed in the input and a capacitor C1 is added in parallel with the feedback resistor (R) to reduce the tendency to oscillate of the circuit.
\end{tasks}
\end{itemize}

\pagebreak