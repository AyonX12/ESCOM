\subsection{Non-Inverting Zero-Crossing Level Detector:}

The circuit that we are going to analyze it's the one in Figure 3.1.0 using the following parameters: \hfill \break

{\bfseries
\begin{itemize}
\item $V_{ent}$ = 16 $V_{pp}$.
\item $V_{ref}$ = 0 V.
\item $V_{sat}$ = 12 V.
\end{itemize}} \hfill

{\bfseries\itshape\color{Violet}{
\begin{itemize}
\item For $V_{sal}$ in the positive semi-cycle:
\end{itemize}}} 

\begin{flushright}
{\bfseries\itshape\color{carmine}{Formula: If $V_{ent}\ >\ V_{ref}$ then $V_{sal}\ =\ V_{sat}$.}} \hfill \break
\end{flushright}

\begin{ceqn}
\begin{align*}
Because:\ 8\ V_{p}\ >\ 0\ V\ then\ V_{sal}\ =\ 12\ V.
\end{align*}
\end{ceqn} \hfill \break

{\bfseries\itshape\color{Violet}{
\begin{itemize}
\item For $V_{sal}$ in the negative semi-cycle:
\end{itemize}}} 

\begin{flushright}
{\bfseries\itshape\color{carmine}{Formula: If $V_{ent}\ <\ V_{ref}$ then $V_{sal}\ =\ -V_{sat}$:}} \hfill \break
\end{flushright}

\begin{ceqn}
\begin{align*}
Because:\ -8\ V_{p}\ <\ 0\ V\ then\ V_{sal}\ =\ -12\ V.
\end{align*}
\end{ceqn} \hfill \break

{\bfseries\itshape\color{carmine}{Finally, we can say that $V_{sal}$ = 24 $V_{pp}$.}} \hfill \break

\pagebreak