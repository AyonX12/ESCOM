\subsection{Inverting Zero-Crossing Level Detector With Hysteresis:}

The circuit that we are going to analyze it's the one in Figure 3.3.0 using the following parameters: \hfill \break

{\bfseries
\begin{itemize}
\item $V_{ent}$ = 16 $V_{pp}$.
\item $V_{sat}$ = 12 V.
\item $R_{1}$ = 2.2 K $\Omega$.
\item $R_{2}$ = 3.9 K $\Omega$.
\end{itemize}} \hfill

{\bfseries\itshape\color{Violet}{
\begin{itemize}
\item For VUS:
\end{itemize}}} 

\begin{flushright}
{\bfseries\itshape\color{carmine}{Formula: VUS = $\frac{(\ V_{sat}\ )(\ R_{1}\ )}{R_{1}\ + R_{2}}$.}} \hfill \break
\end{flushright}

\begin{ceqn}
\begin{align*}
VUS\ &=\ \frac{(\ 12\ V\ )(\ 2.2\ K\ \Omega\ )}{2.2\ K\ \Omega\ + 3.9\ K\ \Omega} \\ \\
&=\ 4.32\ V.
\end{align*}
\end{ceqn} \hfill \break

{\bfseries\itshape\color{Violet}{
\begin{itemize}
\item For VUI:
\end{itemize}}} 

\begin{flushright}
{\bfseries\itshape\color{carmine}{Formula: VUS = $\frac{(\ -V_{sat}\ )(\ R_{1}\ )}{R_{1}\ + R_{2}}$.}} \hfill \break
\end{flushright}

\begin{ceqn}
\begin{align*}
VUS\ &=\ \frac{(\ -12\ V\ )(\ 2.2\ K\ \Omega\ )}{2.2\ K\ \Omega\ + 3.9\ K\ \Omega} \\ \\
&=\ -4.32\ V.
\end{align*}
\end{ceqn} \hfill \break

{\bfseries\itshape\color{Violet}{
\begin{itemize}
\item For the Hysteresis voltage $V_{H}$:
\end{itemize}}} 

\begin{flushright}
{\bfseries\itshape\color{carmine}{Formula: $V_{H}$ = $\abs{VUS}$ + $\abs{VUI}$.}} \hfill \break
\end{flushright}

\begin{ceqn}
\begin{align*}
V_{H}\ &=\ \abs{4.32\ V_{p}}\ +\ \abs{-4.32\ V_{p}} \\ \\
&=\ 8.64\ V.
\end{align*}
\end{ceqn} \hfill \break

\pagebreak