\section{Questionnaire:}

\begin{itemize} 
\item {\bfseries What is the difference between the instrumentation amplifier and the subtracter?} \hfill \break

The main difference is that in the instrumentation amplifier is that the inputs that go to the operational amplifier are connected to each other and in the subtracter are separated. Another of the main differences is that the instrumentation amplifier works with very small voltages such as the micro volt scale \hfill \break

\item {\bfseries Mention 3 examples where the instrumentation amplifier is used:} 

\begin{enumerate}
\item To amplify biological electrical signals (for example in electrocardiograms).
\item As part of circuits to provide constant current power.
\item In power supplies.
\end{enumerate}

\item {\bfseries How is the gain of the instrumentation amplifier calculated?} \hfill \break

It is divided into 2: pre-amplified gain and differential gain, the product of these two mentioned gains is the total gain of the instrumentation amplifier. \hfill \break

\item {\bfseries Which is the purpose to add him a resistance in parallel to the capacitor in the integrador and a capacitor in parallel to the resistance of the derivator?} \hfill \break

Common differential amplifiers are used in circuits where the impedance of the source is low. Strain gauge transducers, for example, typically have elements with lower resistance than 1K$\Omega$.
\end{itemize}

\pagebreak