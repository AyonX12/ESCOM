\section{Questionnaire:}

\begin{itemize} 
\item {\bfseries\itshape What is the reason for the polarization of the transistor?} \hfill \break

For a bipolar transistor to function properly, it is necessary to polarize correctly. For that: 

\begin{tasks}
\task The BASE - EMITTER junction must be directly polarized. 
\task The junction COLLECTOR - BASE must be reverse biased.
\end{tasks} \hfill

\item {\bfseries\itshape What represents the $\beta$ of the transistor?} \hfill \break

The Beta parameter of a bipolar transistor or BJT indicates the efficiency of the transistor, relating
the collector current with the base current, While the $\beta$ it's greater, more efficient the
transistor will be. i.e with a small base current it is capable of delivering a big collector current. Thus, the current gain of a transistor is the relationship that exists between the variation or increase of the collector current and the variation of the base current. Then, $\beta$ it's the gain in the common emitter configuration.  \hfill \break

\item {\bfseries\itshape What does the $\alpha$ of the transistor represent?} \hfill \break

The parameter $\alpha$ of a transistor indicates the similarity relation that occurs in the collector current
and the variations of the emitter currents. Since the base current is usually very small, in most transistors the
parameter $alpha$ approaches the unit. Then, $\alpha$ it's the gain in the common base configuration.  \hfill \break

\item {\bfseries\itshape Mention the points of operation of a transistor:} \hfill \break

Obtaining the working point Q of a device basically consists of obtaining the value of the different
tensions and currents that are established as unknown in the operation the same.  \hfill \break

\item {\bfseries\itshape What is the saturation zone of a bipolar transistor?} \hfill \break

The collector diode is directly polarized and is transistor behaves as a small resistor.
In this zone an additional increase of the base current does not cause an increase of the current of
collector, this depends exclusively on the voltage between emitter and collector. The transistor resembles in its
emitter-collector circuit to a closed switch.  \hfill \break

\item {\bfseries\itshape What is the cut-off zone of a bipolar transistor?} \hfill \break 

The fact of making the base current void is equivalent to keeping the emitter base circuit open,
in these circumstances the collector current is practically null and therefore can be considered the
transistor in its circuit Collector - Emitter as an open circuit breaker.  \hfill \break

\item {\bfseries\itshape What is the difference between the transistor 2N2222 and the TIP41?} \hfill \break

Despite of both are NPN BJT's, the TIP41 it's designed to withstand higher levels of current, voltage and analogously power.  \hfill \break

\item {\bfseries\itshape Mention 3 applications of circuits in commutation:} \hfill \break  

\begin{tasks}
\task Bipolar transistor as switch in ON and OFF states.
\task Bistable multivibrator (FLIP-FLOP).
\task Astable multivibrator.
\end{tasks}

\end{itemize}

\pagebreak