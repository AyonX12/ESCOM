\section{Questionnaire:}

\begin{itemize}
\item {\bfseries\itshape Mention the principle of operation of the zener diode?} \hfill \break 

Zener diodes are diodes that are designed to maintain a constant voltage at their terminals, called Voltage or Zener Voltage (Vz) when reverse polarized, ie when the cathode is with a positive voltage and the negative anode A zener diode in connection with the reverse polarization always has the same voltage at the ends (zener voltage). \hfill \break

\item {\bfseries\itshape What happens to a zener if the voltage of the source is less than its voltage?} \hfill \break

The diode does not conduct voltage, we will only have the constant voltage Vz, when connected to a voltage equal to Vz or greater. \hfill \break

\item {\bfseries\itshape What is the purpose of a voltage regulator?} \hfill \break

A voltage regulator is an electronic device designed to maintain a constant voltage level. \hfill \break

\item {\bfseries\itshape What output voltage do you have on a 5 volt fixed voltage regulator if the input voltage is 5 volts?} \hfill \break

0 volts. \hfill \break

\item {\bfseries\itshape Why in the variable voltage regulators the minimum is 1.2 V?} \hfill \break

It is due to the barrier voltages of some internal semiconductor regulators that are between the "output" pin and the "adjust" pin.
\end{itemize}

\pagebreak