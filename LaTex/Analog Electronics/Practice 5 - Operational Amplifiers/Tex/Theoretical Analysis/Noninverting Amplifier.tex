\subsection{Non-inverting Amplifier:}

The circuit that we are going to analyze it's the one in Figure 3.2.0 using the following parameters: \hfill \break

{\bfseries
\begin{itemize}
\item $V_{ent}$ = 1 $V_{pp}$
\item $R_{f}$ = 10 K $\Omega$
\item R = 1 K $\Omega$
\end{itemize}} \hfill

{\bfseries\itshape\color{Violet}{
\begin{itemize}
\item For $V_{sal}$ in the positive semi-cycle:
\end{itemize}}} 

\begin{flushright}
{\bfseries\itshape\color{carmine}{Formula: $V_{sal}\ =\ (\ \frac{R_{f}}{R}\ +\ 1\ ) \cdot (\ V_{ent}\ )$:}} \hfill \break
\end{flushright}

\begin{ceqn}
\begin{align*}
V_{sal}\ &=\ (\ \frac{10\ K\ \Omega}{1\ K\ \Omega}\ +\ 1\ ) \cdot (\ 0.500\ V_{p}\ ) \\ \\
&=\ 5.5\ V_{p}
\end{align*}
\end{ceqn} \hfill \break

{\bfseries\itshape\color{Violet}{
\begin{itemize}
\item For $V_{sal}$ in the negative semi-cycle:
\end{itemize}}} 

\begin{flushright}
{\bfseries\itshape\color{carmine}{Formula: $V_{sal}\ =\ (\ \frac{R_{f}}{R}\ +\ 1\ ) \cdot (\ V_{ent}\ )$:}} \hfill \break
\end{flushright}

\begin{ceqn}
\begin{align*}
V_{sal}\ &=\ (\ \frac{10\ K\ \Omega}{1\ K\ \Omega}\ +\ 1\ ) \cdot (\ -\ 0.500\ V_{p}\ ) \\ \\
&=\ -5.5\ V_{p}
\end{align*}
\end{ceqn} \hfill \break

{\bfseries\itshape\color{Violet}{
\begin{itemize}
\item For the Gain:
\end{itemize}}} 

\begin{flushright}
{\bfseries\itshape\color{carmine}{Formula: $A_{v}\ =\ (\ \frac{R_{f}}{R}\ +\ 1\ )$:}} \hfill \break
\end{flushright}

\begin{ceqn}
\begin{align*}
A_{v}\ &=\ (\ \frac{10\ K\ \Omega}{1\ K\ \Omega}\ +\ 1\ ) \\ \\
&=\ 11
\end{align*}
\end{ceqn} \hfill \break

{\bfseries\itshape\color{carmine}{Finally we can assume that $V_{sal}$ = 11 $V_{pp}$.}} \hfill \break

\pagebreak